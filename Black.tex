\documentclass[11pt,a4paper]{article}
\usepackage[margin=1in]{geometry}
\usepackage{amsmath,amsthm,amssymb}
\usepackage{mathtools}
\usepackage{xcolor}
\usepackage{fancyhdr}
\usepackage{titlesec}
\usepackage{framed}

% Custom colorslack
\definecolor{theoremblue}{RGB}{0,102,204}
\definecolor{examplegreen}{RGB}{0,128,0}
\definecolor{remarkred}{RGB}{204,0,0}

% Theorem environments
\newtheorem{theorem}{Theorem}[section]
\newtheorem{lemma}[theorem]{Lemma}
\newtheorem{definition}[theorem]{Definition}

% Simple colored boxes using framed
\newenvironment{keybox}[1][Key Result]
{\begin{leftbar}\noindent\textbf{\color{theoremblue}#1}\par\vspace{0.5em}}
{\end{leftbar}}

\newenvironment{remarkbox}[1][Remark]
{\begin{leftbar}\noindent\textbf{\color{remarkred}#1}\par\vspace{0.5em}}
{\end{leftbar}}

\newenvironment{examplebox}[1][Example]
{\begin{leftbar}\noindent\textbf{\color{examplegreen}#1}\par\vspace{0.5em}}
{\end{leftbar}}

% Header and footer
\pagestyle{fancy}
\fancyhf{}
\rhead{Feynman-Kac \& Black-Scholes}
\lhead{\leftmark}
\cfoot{\thepage}

% Title formatting
\titleformat{\section}{\Large\bfseries\color{theoremblue}}{\thesection}{1em}{}
\titleformat{\subsection}{\large\bfseries\color{theoremblue}}{\thesubsection}{1em}{}

\title{\textbf{\Large The Feynman-Kac Theorem and\\Black-Scholes Formula Derivation}}
\author{Dean Palermo}
\date{\today}

\begin{document}

\maketitle

\begin{abstract}
This document provides a comprehensive derivation of the Black-Scholes formula using the Feynman-Kac theorem. We begin with the fundamental partial differential equation (PDE) governing option pricing and demonstrate how stochastic calculus transforms this into a probabilistic expectation. The derivation culminates in the famous Black-Scholes formula for European call options.
\end{abstract}

\tableofcontents
\newpage

\section{The Black-Scholes Partial Differential Equation}

The Black-Scholes equation for the price $f(t,S)$ of a derivative security is given by:

\begin{keybox}[The Black-Scholes PDE]
\begin{equation}
\frac{\partial f}{\partial t} + \frac{1}{2}\sigma^2 S^2 \frac{\partial^2 f}{\partial S^2} + rS \frac{\partial f}{\partial S} - rf = 0
\end{equation}
with boundary condition: $f(T,S) = \Phi(S)$ (payoff function)
\end{keybox}

where:
\begin{itemize}
    \item $S$ is the stock price
    \item $t$ is time
    \item $r$ is the risk-free interest rate
    \item $\sigma$ is the volatility
    \item $T$ is the maturity time
    \item $\Phi(S)$ is the payoff function at maturity
\end{itemize}

\section{The Feynman-Kac Theorem}

The Feynman-Kac theorem provides a stochastic representation for solutions to certain parabolic PDEs.

\begin{theorem}[Feynman-Kac Theorem]
Consider the PDE:
\begin{equation}
\frac{\partial v}{\partial t} + \frac{1}{2}a^2(x) \frac{\partial^2 v}{\partial x^2} + b(x) \frac{\partial v}{\partial x} - rv = 0
\end{equation}
with terminal condition $v(T,x) = g(x)$.

If $X_t$ satisfies the SDE:
\begin{equation}
dX_s = b(X_s)ds + a(X_s)dW_s
\end{equation}

Then the solution is:
\begin{equation}
v(t,x) = \mathbb{E}\left[e^{-r(T-t)}g(X_T) \mid X_t = x\right]
\end{equation}
\end{theorem}

\subsection{Intuitive Understanding}

\begin{examplebox}[Physical Interpretation]
Think of $v(t,x)$ as representing a particle starting at position $x$ at time $t$:
\begin{enumerate}
    \item The particle evolves according to the stochastic process $dX_s = b\,ds + a\,dW_s$
    \item At time $T$, apply function $g$ to get $g(X_T)$
    \item Discount back to present value: $e^{-r(T-t)}g(X_T)$
    \item Take expectation: this gives us $v(t,x)$
\end{enumerate}
\end{examplebox}

\begin{remarkbox}[What does "discount back" mean?]
The factor $e^{-r(T-t)}$ is the continuous compounding discount factor. If you're promised $g(X_T)$ dollars at time $T$, its present value at time $t$ is $e^{-r(T-t)}g(X_T)$.
\end{remarkbox}

\section{Deriving the Feynman-Kac Formula}

\subsection{Setting up the Martingale Property}

From the definition of $v(t,x)$, we have:
\begin{equation}
v(t,x) = \mathbb{E}\left[e^{-r \cdot dt} \cdot v(t+dt, X_{t+dt}) \mid X_t = x\right]
\end{equation}

Using the approximation $e^{-r \cdot dt} \approx 1 - r \cdot dt$ for small $dt$:
\begin{equation}
v(t,x) = \mathbb{E}[(1 - r \cdot dt)(v(t,x) + dv)]
\end{equation}

\subsection{Applying Itô's Formula}

For the process $X_t$ satisfying $dX = b\,dt + a\,dW$, Itô's formula gives:
\begin{align}
dv &= \frac{\partial v}{\partial t}dt + \frac{\partial v}{\partial x}dX + \frac{1}{2}\frac{\partial^2 v}{\partial x^2}(dX)^2\\
&= \left[\frac{\partial v}{\partial t} + b\frac{\partial v}{\partial x} + \frac{1}{2}a^2\frac{\partial^2 v}{\partial x^2}\right]dt + a\frac{\partial v}{\partial x}dW
\end{align}

Taking expectations (noting $\mathbb{E}[dW] = 0$):
\begin{equation}
\mathbb{E}[dv] = \left[\frac{\partial v}{\partial t} + b\frac{\partial v}{\partial x} + \frac{1}{2}a^2\frac{\partial^2 v}{\partial x^2}\right]dt
\end{equation}

\subsection{Deriving the PDE}

From the martingale property and ignoring higher-order terms:
\begin{equation}
0 = \mathbb{E}[dv] - r \cdot dt \cdot v(t,x)
\end{equation}

Substituting and dividing by $dt$:
\begin{keybox}[The Feynman-Kac PDE]
\begin{equation}
\frac{\partial v}{\partial t} + \frac{1}{2}a^2\frac{\partial^2 v}{\partial x^2} + b\frac{\partial v}{\partial x} - rv = 0
\end{equation}
\end{keybox}

\section{Application to Black-Scholes}

\subsection{Matching the Pattern}

For the Black-Scholes equation:
\begin{equation}
\frac{\partial f}{\partial t} + \frac{1}{2}\sigma^2 S^2 \frac{\partial^2 f}{\partial S^2} + rS \frac{\partial f}{\partial S} - rf = 0
\end{equation}

We identify:
\begin{itemize}
    \item $x = S$ (stock price)
    \item $a(S) = \sigma S$ (diffusion coefficient)
    \item $b(S) = rS$ (drift coefficient)
    \item Discount rate = $r$
    \item Terminal condition: $g(S) = \Phi(S)$
\end{itemize}

\subsection{The Risk-Neutral Process}

The corresponding stochastic process is:
\begin{keybox}[Risk-Neutral Stock Price Process]
\begin{equation}
dS_s = rS_s \, ds + \sigma S_s \, dW_s
\end{equation}
\end{keybox}

This represents the stock price evolution under the risk-neutral measure, where the drift is the risk-free rate $r$ (not the actual expected return $\mu$).

\subsection{The Solution Formula}

By Feynman-Kac, the option price is:
\begin{keybox}[Black-Scholes Solution]
\begin{equation}
f(t,S) = \mathbb{E}^{\mathbb{Q}}\left[e^{-r(T-t)}\Phi(S_T) \mid S_t = S\right]
\end{equation}
where $\mathbb{Q}$ is the risk-neutral measure.
\end{keybox}

\section{Solving for Geometric Brownian Motion}

\subsection{Using Itô's Formula on $\ln(S)$}

Let $Y_s = \ln(S_s)$. Applying Itô's formula with $f(S) = \ln(S)$:
\begin{align}
\frac{\partial f}{\partial S} &= \frac{1}{S}\\
\frac{\partial^2 f}{\partial S^2} &= -\frac{1}{S^2}
\end{align}

\begin{align}
d(\ln S_s) &= \frac{1}{S_s}dS_s + \frac{1}{2}\left(-\frac{1}{S_s^2}\right)(dS_s)^2\\
&= \frac{1}{S_s}(rS_s \, ds + \sigma S_s \, dW_s) - \frac{1}{2S_s^2}\sigma^2 S_s^2 \, ds\\
&= r \, ds + \sigma \, dW_s - \frac{1}{2}\sigma^2 \, ds\\
&= \left(r - \frac{1}{2}\sigma^2\right)ds + \sigma \, dW_s
\end{align}

\subsection{The Explicit Solution}

Integrating from $t$ to $T$:
\begin{equation}
\ln(S_T) - \ln(S_t) = \left(r - \frac{1}{2}\sigma^2\right)(T-t) + \sigma(W_T - W_t)
\end{equation}

Since $W_T - W_t \sim N(0, T-t)$, we can write $W_T - W_t = \sqrt{T-t} \cdot Z$ where $Z \sim N(0,1)$:

\begin{keybox}[Stock Price at Maturity]
\begin{equation}
S_T = S_t \exp\left[\left(r - \frac{1}{2}\sigma^2\right)(T-t) + \sigma\sqrt{T-t} \cdot Z\right]
\end{equation}
where $Z \sim N(0,1)$.
\end{keybox}

\section{The European Call Option}

\subsection{Setting up the Expectation}

For a European call option with strike $K$, the payoff is $\Phi(S_T) = \max(S_T - K, 0)$.

The option value is:
\begin{equation}
C(t,S_t) = e^{-r(T-t)}\mathbb{E}^{\mathbb{Q}}[\max(S_T - K, 0)]
\end{equation}

\subsection{Splitting the Expectation}

Using the indicator function representation:
\begin{align}
\max(S_T - K, 0) &= (S_T - K) \cdot \mathbf{1}_{S_T > K}\\
\mathbb{E}[\max(S_T - K, 0)] &= \mathbb{E}[S_T \cdot \mathbf{1}_{S_T > K}] - K \cdot \mathbb{E}[\mathbf{1}_{S_T > K}]
\end{align}

We need to evaluate:
\begin{enumerate}
    \item $\mathbb{E}[\mathbf{1}_{S_T > K}] = \mathbb{P}(S_T > K)$ — probability of exercise
    \item $\mathbb{E}[S_T \cdot \mathbf{1}_{S_T > K}]$ — expected stock value if exercised
\end{enumerate}

\subsection{Finding the Critical Values}

The condition $S_T > K$ is equivalent to:
\begin{align}
S_t \exp\left[\left(r - \frac{1}{2}\sigma^2\right)(T-t) + \sigma\sqrt{T-t} \cdot Z\right] &> K\\
\left(r - \frac{1}{2}\sigma^2\right)(T-t) + \sigma\sqrt{T-t} \cdot Z &> \ln\left(\frac{K}{S_t}\right)\\
Z &> \frac{\ln(K/S_t) - (r - \frac{1}{2}\sigma^2)(T-t)}{\sigma\sqrt{T-t}}
\end{align}

\begin{keybox}[Critical Values]
Define:
\begin{align}
d_2 &= \frac{\ln(S_t/K) + (r - \frac{1}{2}\sigma^2)(T-t)}{\sigma\sqrt{T-t}}\\
d_1 &= d_2 + \sigma\sqrt{T-t} = \frac{\ln(S_t/K) + (r + \frac{1}{2}\sigma^2)(T-t)}{\sigma\sqrt{T-t}}
\end{align}
\end{keybox}

Then $S_T > K$ when $Z > -d_2$, or equivalently, $Z < d_2$.

\subsection{Evaluating the Integrals}

\textbf{First integral:}
\begin{equation}
\mathbb{E}[\mathbf{1}_{S_T > K}] = \mathbb{P}(Z > -d_2) = \mathbb{P}(Z < d_2) = N(d_2)
\end{equation}

\textbf{Second integral:} Using a measure change technique (completing the square), we get:
\begin{equation}
\mathbb{E}[S_T \cdot \mathbf{1}_{S_T > K}] = S_t e^{r(T-t)} N(d_1)
\end{equation}

\subsection{The Final Formula}

Combining everything:
\begin{align}
C(t,S_t) &= e^{-r(T-t)}[S_t e^{r(T-t)} N(d_1) - K N(d_2)]\\
&= S_t N(d_1) - Ke^{-r(T-t)} N(d_2)
\end{align}

\begin{keybox}[Black-Scholes Formula for European Call]
\begin{equation}
\boxed{C(t,S) = S N(d_1) - Ke^{-r(T-t)} N(d_2)}
\end{equation}
where:
\begin{align}
d_1 &= \frac{\ln(S/K) + (r + \frac{1}{2}\sigma^2)(T-t)}{\sigma\sqrt{T-t}}\\
d_2 &= \frac{\ln(S/K) + (r - \frac{1}{2}\sigma^2)(T-t)}{\sigma\sqrt{T-t}} = d_1 - \sigma\sqrt{T-t}
\end{align}
and $N(\cdot)$ is the cumulative standard normal distribution function.
\end{keybox}

\section{Economic Interpretation}

\begin{examplebox}[Understanding the Terms]
\begin{itemize}
    \item \textbf{$S N(d_1)$}: Expected stock value if the option is exercised, weighted by the hedge ratio
    \item \textbf{$Ke^{-r(T-t)} N(d_2)$}: Present value of the strike price, weighted by the risk-neutral probability of exercise
    \item \textbf{$N(d_2)$}: Risk-neutral probability that $S_T > K$ (option finishes in-the-money)
    \item \textbf{$N(d_1)$}: The option's delta ($\partial C/\partial S$), representing the hedge ratio
\end{itemize}
\end{examplebox}

\section{Conclusion}

The Feynman-Kac theorem provides an elegant bridge between the world of partial differential equations and stochastic processes. By transforming the Black-Scholes PDE into a probabilistic expectation, we can derive the famous Black-Scholes formula through careful application of stochastic calculus and measure theory.

This approach not only yields the formula but also provides deep economic insights into option pricing, revealing the option price as an expected discounted payoff under the risk-neutral measure.

\end{document}